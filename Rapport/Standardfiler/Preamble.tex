\documentclass[11pt,a4paper]{\docklasse}	% Vælg article eller report
\NeedsTeXFormat{LaTeX2e}[1994/12/01]		% Angiver hvilken LaTeX distribution, der er minimum
\usepackage[utf8]{inputenc}					% Giver adgang til specialtegn som æ, ø og å
\usepackage[T1]{fontenc}					% Giver outputformateringen af skriften
\RequirePackage{ifthen}
\newcommand{\hvisdansk}[2]{\ifthenelse{\equal{\danish}{true}}{#1}{#2}}
\def\ifdanish{\equal{\danish}{true}}
\hvisdansk%
{\usepackage[english,danish]{babel}}%		% Dansk sprogformatering som 1. prioritet, dernæst engelsk
{\usepackage[danish,main=UKenglish]{babel}}				% Afsnitsbetegnelse, datoformatering  osv.
											% english bruges til bl.a. \mathbb{}
% Under Options>Configure Texmaker>Commands skal "Makeindex" feltet ændres til "makeindex %.nlo -s nomencl.ist -o %.nls -t %.nlg" (Dette for at kunne bruge nomenclature)
% For renere/mere overskuelige filmapper skal der sættes flueben i "Use "build" subdirectory for output files
% Når dette flueben er sat skal man også ændre skrives "build\" foran alle %-tegnene under "Bib(la)tex" og "Makeindex"


% ==================================================================
% ============== Generelt ==========================================
% ==================================================================

\usepackage{forloop}		% Giver mulighed for at lave forløkker

\usepackage{totcount}		% Giver mulighed for at referere til counterens sidste værdi
\newtotcounter{BilagSider}	% Bruges til informationssiden
\usepackage[inline]{enumitem} % inline enum list

% Farver
\usepackage{color}

% Manuel fastsættelse af floatnummereringens dybde
%\numberwithin{figure}{chapter}
%\numberwithin{equation}{section}
%\numberwithin{table}{chapter}

% Opsætning af linjeafstand
%\usepackage{setspace}		% Derefter kan der løbende kaldes \singlespacing, \onehalfspacing eller \doublespacing
							% Disse kan også kaldes i miljøer, f.eks. \begin{onehalfspace}, \begin{spacing}{2.5}

% For mere kontrol kan linjeafstand og andre afstande opsættes manuelt
% Det er muligt at definere afstandende med plus og/eller minus margin <længde> plus <ekstra længde> minus <mindre længde>
%\setlength{\baselineskip}{<længde>}	% Linjeafstand
%\setlength{\parskip}{<længde>}			% Afsnitsafstand
%\setlength{\parindent}{<længde>}		% Indrykning af første linje

% (Obs. den aktuelle afstand kan printes ved at skrive \the\<afstandsparameter>

% Nummerering af subsubsection
\setcounter{secnumdepth}{3}
\renewcommand{\thesubsubsection}{\thesubsection .\arabic{subsubsection}}
%\setcounter{tocdepth}{3}		% Denne kan indkommenteres, hvis subsubsections ønskes i indholdsfortegnelsen

%\usepackage[all]{nowidow}		% Kan fjerne horeunger, hvis de er et problem

% ==================================================================
% ============== Referencer og links ===============================
% ==================================================================
% Hyperref laver referencer om til linkede referencer og hyperlink samt laver bogmærker
% Brug \href{<url>}{<Tekst der printes>} til at lave hyperlinks
% Der kan refereres til et tilfældigt stykke tekst vha. \hypertarget{<label>}{<target caption>} og \hyperlink{<label>}{<link caption>}

\PassOptionsToPackage{hyphens}{url}\usepackage[	% \PassOptionsToPackage{hyphens}{url} tillader line-breaking af lange url'er
			bookmarks=true			% Viser bookmarks når pdf'en åbnes
			,colorlinks=true		% true=farvede links, false(default)=farvede rammer
			,linkcolor=red
			,citecolor=green
			,filecolor=magenta
			,urlcolor=blue
			,hidelinks				% Fjerner rammer og farve (udkommenteres for at aktivere farvede links)
			,bookmarksnumbered
			,breaklinks=true
			]{hyperref}
\usepackage{bookmark}
			

%\newcommand{\tref}[2]				% \tref{Test}{eq:lign} vil give outputtet Test 2 (hvis eq:lign refererer til ligning 2), hvor både "Test" og "2" er links.
%		{\hyperref[#2]{#1~\ref*{#2}}}
	

%\usepackage{cleveref}				% Alternativ referencepakke. Kan bl.a. bruges til \crefrange{<1. label>}{<2. label>}, men den kræver opsætning.



% ==================================================================
% ============== Marginer og headings ==============================
% ==================================================================
% Marginer og sidehoved- / -fodstørrelse
\usepackage	[a4paper,
			includeheadfoot,
			head=1.8cm,
			foot=.5cm,
			headsep=.5cm,
			footskip=1cm,
			bindingoffset=0cm,
			top=.7cm,
			bottom=2cm,
			inner=2.5cm,
			outer=2.5cm,
			marginparsep=0.5cm,
			marginparwidth=2.4cm
			]{geometry}	

\savegeometry{A4ligemargen}		% Gemmer indstillinger til forside

\geometry{inner=2.5cm,
		  outer=3.5cm}
\savegeometry{A4uligemargen}	% Gemmerindstillinger for uens margen

\geometry{paperwidth=420mm}
\savegeometry{A3}				% Gemmerindstillinger for bredere side
\geometry{paperwidth=590mm}
\savegeometry{A4x3}				% Gemmerindstillinger for endnu bredere side

\loadgeometry{A4uligemargen}	% Lige-/uligemargen samt A3/A4x3 kan hentes med \loadgeometry{<>}
								% OBS. kaldes A3 eller A4x3 skal der først kaldes
								% \eject \pdfpagewidth=<420 eller 590>mm \pdfpageheight=297mm
								% \titlespacing*{\chapter}{1cm}{0cm}{1cm}[<22 eller 39>cm]
								% Når man derefter skifter tilbage skal der nulstilles til
								% \eject \pdfpagewidth=210mm \pdfpageheight=297mm
								% \titlespacing*{\chapter}{1cm}{0cm}{1cm}[1cm]

%\usepackage{marginnote}		% Bruges hvis marginnoter skal placeres i en bestemt højde eller i et float-miljø.
								% Syntaks: \marginnote{<Tekst m.m.>}[<lodret offset>]

\newcommand{\blankside}[1][]{%	% Kodeblok til at lave en blank bagside på 2-sidet format
			\thispagestyle{empty}
			\ 
			\vspace{9cm}
			\begin{center}
			#1
			\end{center}
			\clearpage
			\thispagestyle{fancy}}

% Indhold i sidehoved / -fod
\usepackage{lastpage}			% Bruges til at referere til sidste sidenummer
\usepackage{fancyhdr}
	\pagestyle{fancy}
% Sidehoved
	\fancyhead[LO,RE]{}			% Indvendig
	\fancyhead[CO,CE]{}			% Midt
	\fancyhead[RO,LE]{}			% Udvendig		
% Sidefod
	% Indvendig% \leftmark kan ændres til \rightmark, hvis der ønskes sectionnavne
\newcommand{\StandardFooter}{
	\fancyhead[LO]{\parbox[t][0mm][t]{11cm}{\raggedright\leftmark}}
	\fancyhead[RE]{\parbox[t][0mm][t]{11cm}{\raggedleft\leftmark}}
	\fancyfoot[CO,CE]{}								% Midt
	\fancyfoot[RO,LE]{\ifthenelse{\ifdanish}{Side \thepage\ af \pageref{LastPage}}{Page \thepage\ of \pageref{LastPage}}}	% Udvendig
}
\StandardFooter
% Hvis der vælges 1-sidet format, kan sidehovedet/-fod kaldes med \chead{}, \rhead{}, \lhead{}, \cfoot{}, \rfoot{}, \lfoot{}	

% Streger
	\renewcommand{\headrulewidth}{0.5pt}	% Tykkelse af topstreg
	\renewcommand{\footrulewidth}{0.5pt}	% Tykkelse af bundstreg

	\fancypagestyle{plain}		% Lav fancyheader på sider med chapter
	
% Dermed kan billeder indsættes (umiddelbart efter section-/chaptertitlen) vha. f.eks.
% \chead{\includegraphics[height=1.7cm]{<filnavn>}
% OBS! Med disse opsætninger må figurhøjden ikke overstige 1.7cm.


% ==================================================================
% ============== Chapterformatering ================================
% ==================================================================	
\usepackage{titlesec}			%Muliggør formatering af chaptertitler, sectionstitler osv.

\titleformat{\chapter}
			[frame]	% Kapitel type
			{\normalfont}				% Generel formatering
			{\filright					% Venstrejusteret
				\footnotesize			% Tekststørrelse af kapitellabel
				\enspace				% Afstand før kapitellabel
				\chaptername~\thechapter		% Kapitellabel
				\enspace}				% Afstand efter kapitellabel
			{8pt}						% Afstand mellem ramme og titel
			{\Large\bfseries\filcenter}	% Titelformatering

\titlespacing*{\chapter}		% Afstand omkring chaptertitler (* fjerne indrykning på første linje)
  {1cm}{0cm}{1cm}[1cm]			% {venste}{før}{efter}[højre}
  
  
% ==================================================================
% ============== Bilag =============================================
% ==================================================================
\newcommand{\startbilag}{		% Kommandoen kaldes i main og opsætter formateringen af bilag
	%\clearpage
	\appendix
	\ifthenelse{\equal{\docklasse}{report}}
		{\hvisdansk%
			{\renewcommand{\chaptername}{Bilag}}%
			{\renewcommand{\chaptername}{Appendix}}%
		}%
		{}
	%\clearpage
	\ifthenelse{\equal{\docklasse}{report}}%
		{\hvisdansk%
			{\addcontentsline{toc}{chapter}{Bilag}}%
			{\addcontentsline{toc}{chapter}{Appendices}}}%
		{\phantomsection\hvisdansk%
			{\addcontentsline{toc}{section}{Bilag}}%
			{\addcontentsline{toc}{section}{Appendices}}}%
	\chead{}
	\hvisdansk{	
	\fancyhead[RO]{\raggedleft BILAG}\fancyhead[LE]{\raggedright BILAG}}{
	\fancyhead[RO]{\raggedleft APPENDIX}\fancyhead[LE]{\raggedright APPENDIX}}
	}
%
% ==================================================================
% ============== Ord- og symbolliste ===============================
% ==================================================================
% For at anvende disse skal man først compile (F6) dernæst MakeIndex (F12) og så compile igen
% Syntaks: \nm[<gruppe>]{<Ord/Symbol>}{<Forklaring\nomunit[<værdi>]{<enhed>}>}
%	gruppe: O for ord, K for konstant, V for variabel
%	Ord/Symbol: Ord/Symbol der skal i listen
%	Forklaring: Kort forklaring af ordet/symbolet evt. med enheder
%	\nomunit kan kaldes hvis der skal enhed på konstanter og variable. <værdi> er konstanters værdi og er optional

% Brug \printnomenclature til at oprette symbollisten 
\hvisdansk%
		{\usepackage[danish,							% Dansk syntakt
%					intoc,								% I indholdsfortegnelse (Ønskes der nummereret kapitel indsættes i stedet nedenstående kodeblok i dokumentet)
					refpage								% Tilføjer sidereferencer
					]{nomencl}}%
		{\usepackage[english,							% Engelsk syntakt
%					intoc,								% I indholdsfortegnelse (Ønskes der nummereret kapitel indsættes i stedet nedenstående kodeblok i dokumentet)
					refpage								% Tilføjer sidereferencer
					]{nomencl}}			
							
			
% OBS!!! For at bruge denne skal MakeIndex-feltet under Commands i Configure Texmaker ændres til
% makeindex %.nlo -s nomencl.ist -o %.nls -t %.nlg

% Kodeblok til nummereret kapitel (Indsættes i stedet for \printnomenclature)
%\let\stdchapter\chapter
%\def\chapter*#1{\stdchapter{#1}}
%\printnomenclature
%\let\chapter\stdchapter

% Aktiver ordliste
\makenomenclature

% Listens navn	
\hvisdansk{\renewcommand{\nomname}{Symbolliste og ordforklaring}}{\renewcommand{\nomname}{Symbols and Nomenclature}}

% Linjeafstand
\setlength{\nomitemsep}{-1pt}

% Der oprettes en ny og kortere kommando \nm der også tjekker om der er tale om et ord
\RequirePackage{scrextend}
\newcommand{\nm}[3][]{
	\ifthenelse{\equal{#1}{O}}
		{\marginpar{\ifthispageodd{}{\flushright}\vspace{-.5\baselineskip}\textbf{\footnotesize #2}}\hspace{-.35em}\nomenclature[C]{#2:}{#3}}%		% Hvis det er et ord, skrives ordet også i marginen
	{\ifthenelse{\equal{#1}{V}}{\nomenclature[A]{#2}{#3}}
	{\ifthenelse{\equal{#1}{S}}{\nomenclature[B]{#2}{#3}}
	{\nomenclature[#1]{#2}{#3}}}}}											% Ønskes dette ikke, kan det erstattes med \newcommand{\nm}[3][*]{\nomenclature[#1]{#2}{#3}}

% Gruppering
	\RequirePackage{ifthen}
	\renewcommand{\nomgroup}[1]{
		\item[]\hspace*{-\leftmargin}							% Fjerner indrykning
		\rule[2pt]{0.43\linewidth}{1pt}\hfill					% Streg før
		 	\ifthenelse{\equal{#1}{A}}{\parbox[c][10mm][c]{0.2\linewidth}{\centering\textbf{\hvisdansk{Konstanter og variabler}{Constants \& Variables}}}}{	% 1. gruppes overskrift
 		 	\ifthenelse{\equal{#1}{B}}{\textbf{\hvisdansk{Andre symboler}{Other Symbols}}}{% 3. gruppes overskrift
 		 	\ifthenelse{\equal{#1}{C}}{\textbf{\hvisdansk{Ordforklaring}{Nomenclature}}}{			% 4. gruppes overskrift
 			\textbf{Fejl index}}}}								% Default gruppes overskrift
		\hfill\rule[2pt]{0.43\linewidth}{1pt}}					% Streg efter

% Enheder. Indsættes til sidst i symboldefinitionen
% Syntaks: \nomunit[<værdi af konstant>]{<enheder>}
% Skal ikke skrives i mathmode
	\newcommand{\nomunit}[2][]{%
		\renewcommand{\nomentryend}{\hspace*{\fill} \footnotesize \ensuremath{\textstyle #1\,\mathrm{\left[#2\right]}}}}


% ==================================================================
% ============== Kildekode =========================================
% ==================================================================
% Denne pakke bruges til at opsætte og genkende kildekode.
% Pakken genkender selv kommentarer og kommandoer, og giver det særlig formatering.
% Manglende kommandoer kan tilføjes og ligeledes kan forkerte kommandoer fjernes
% Koden kan indsættes manuelt i et lstlisting miljø, som inline kode vha. \lstline!<kode>! eller hentes ind fra en fil vha.
% \lstinputlisting[language=Python, firstline=37, lastline=45]{source_filename.py}

% Der defineres tre farve, der bruges til formatering af koden.
\definecolor{mygreen}{RGB}{0,110,0}
\definecolor{mygray}{RGB}{100,100,100}
\definecolor{myorange}{RGB}{205,130,18}
\definecolor{mymauve}{RGB}{148,0,209}
\usepackage{listings}
\lstset{ %				% Opsætning af kodeformateringen. De første options er de vigtige.
	language = [visual]C++,
	alsolanguage = C,
	alsolanguage = [x86masm]Assembler,
	alsolanguage = MuPAD,
	alsolanguage = Matlab,
	alsolanguage = TeX,
%	defaultdialect = [LaTeX]TeX,
%	defaultdialect = [x86masm]Assembler,
%	defaultdialect = [ANSI]C,
%	defaultdialect = [Visual]C++,
%	morekeywords={*,...},            	% if you want to add more keywords to the set
%	deletekeywords={...},            	% if you want to delete keywords from the given language
	basicstyle=\footnotesize\ttfamily,        	% the size of the fonts that are used for the code
	tabsize=4,                       	% sets default tabsize to 3 spaces
	commentstyle=\color{mygreen},    	% comment style
	stringstyle=\color{myorange},    	% string literal style
	keywordstyle=\color{blue},       	% keyword style
%	identifierstyle=,					% variablers formatering
	numberstyle=\tiny\color{mygray}\rmfamily,	% the style that is used for the line-numbers
%	numberbychapter=false,				
	numbers=left,                    	% where to put the line-numbers; possible values are (none, left, right)
	numbersep=5pt,                   	% how far the line-numbers are from the code
	stepnumber=1,                   	% the step between two line-numbers. If it's 1, each line will be numbered
%	title=\lstname                  	% show the filename of files included with \lstinputlisting; also try caption instead of title
%	frame=single,                   	% adds a frame around the code kan kaldes med andre parametre
%	rulecolor=\color{black},        	% if not set, the frame-color may be changed on line-breaks within not-black text (e.g. comments (green here))
%	backgroundcolor=\color{white},  	% choose the background color; you must add \usepackage{color} or \usepackage{xcolor}
%	breakatwhitespace=false,        	% sets if automatic breaks should only happen at whitespace
	breaklines=true,                	% sets automatic line breaking
	captionpos=b,                   	% sets the caption-position to bottom
%	escapeinside={\%*}{*)},         	% if you want to add LaTeX within your code
%	extendedchars=true,             	% lets you use non-ASCII characters; for 8-bits encodings only, does not work with UTF-8
%	keepspaces=true,                 	% keeps spaces in text, useful for keeping indentation of code (possibly needs columns=flexible)
%	showlines=true,					 	% doesnt drop empty lines in the end
%	emptylines=0,						% Maximum empty lines in a row
%	showspaces=true,               		% show spaces everywhere adding particular underscores; it overrides 'showstringspaces'
	showstringspaces=false,         	% underline spaces within strings only
%	showtabs=false,                  	% show tabs within strings adding particular underscores
	abovecaptionskip=1\baselineskip,	% Afstand over caption
	belowcaptionskip=.5\baselineskip,	% Afstand under caption
	aboveskip=1\baselineskip,			% Afstand før kode
	belowskip=1\baselineskip			% Afstand efter kode (er vigtig, hvis der ikke er nogen caption
	}

% Ønskes bestemt linjenummer kan miljøet kaldes med option [firstnumber=<tal>] eller [firstnumber=last].
% last fortsætter, hvor det sidste miljø slap. Der kan også bruges auto kombineret med name (læs i dokumentation).

\renewcommand{\lstlistingname}{Kode}	% Caption label

\lstset{literate=					% Håndterer specialtegn
   {á}{{\'a}}1 {é}{{\'e}}1 {í}{{\'i}}1 {ó}{{\'o}}1 {ú}{{\'u}}1
   {Á}{{\'A}}1 {É}{{\'E}}1 {Í}{{\'I}}1 {Ó}{{\'O}}1 {Ú}{{\'U}}1
   {à}{{\`a}}1 {è}{{\`e}}1 {ì}{{\`i}}1 {ò}{{\`o}}1 {ù}{{\`u}}1
   {À}{{\`A}}1 {È}{{\'E}}1 {Ì}{{\`I}}1 {Ò}{{\`O}}1 {Ù}{{\`U}}1
   {ä}{{\"a}}1 {ë}{{\"e}}1 {ï}{{\"i}}1 {ö}{{\"o}}1 {ü}{{\"u}}1
   {Ä}{{\"A}}1 {Ë}{{\"E}}1 {Ï}{{\"I}}1 {Ö}{{\"O}}1 {Ü}{{\"U}}1
   {â}{{\^a}}1 {ê}{{\^e}}1 {î}{{\^i}}1 {ô}{{\^o}}1 {û}{{\^u}}1
   {Â}{{\^A}}1 {Ê}{{\^E}}1 {Î}{{\^I}}1 {Ô}{{\^O}}1 {Û}{{\^U}}1
   {œ}{{\oe}}1 {Œ}{{\OE}}1 {æ}{{\ae}}1 {Æ}{{\AE}}1 {ß}{{\ss}}1
   {ç}{{\c c}}1 {Ç}{{\c C}}1 {ø}{{\o}}1 {å}{{\r a}}1 {Å}{{\r A}}1
   {€}{{\EUR}}1 {£}{{\pounds}}1
}

% ==================================================================
% ============== Ligninger =========================================
% ==================================================================
\usepackage{amsmath}			% Bruges med align og alignat miljøerne
%\usepackage{mathtools}
\usepackage{amsfonts}				% Diverse specialtegn og \mathbb{•} der kan bruges til at skrive symbol for talsæt (reelle, komplekse, naturlige osv.)

									% Hvis inline math ikke vises rigtigt kan der skrives $\displaystyle <ligning>$
									% Der kan bruges \dfrac{}{} eller \tfrac{}{} til at tvinge displaystyle eller textstyle brøker
									% Der kan bruges \cfrac{}{} hvis der skal nestes flere brøker.
\usepackage{xfrac}					% Bruges til at lave skrå brøker med \sfrac{}{}

\hvisdansk{\usepackage{icomma}}{}					% Til decimalkomma. Laver kun mellemrum efter komma, hvis man selv laver et mellemrum

%\usepackage{amssymb}				% Diverse specialtegn
%\usepackage{bbm}					% Diverse specialtegn

%\usepackage{cancel}				% Vis at ting går ud med hinanden

% Indsætter et lille mellemrum efter \sqrt
\let\oldsqrt\sqrt					% Omdefinerer gammel kvadratrod
\renewcommand{\sqrt}[2][]{\oldsqrt[#1]{#2}\,}	% Definerer ny kvadratrod

% Hvis man vil give en ligning et bestemt nummer (eller tekst) kan man skrive \tag{<indhold>}
% Dette er særlig brugbart, hvis man gentager en ligning. Så kan <indhold> være en reference til den ligning man gentager

\usepackage{gensymb}				%for at kunne lave grader tegn

\newcommand{\unit}[1]{\ensuremath{\hspace{.5ex} \mathrm{#1}}}	% Korrekt formatering af enheder. \unit{<enhed>}
\newcommand{\diff}{\ensuremath{\mathrm{d}}}		% Korrekt formatering af differential d. \diff 
\newcommand{\ddiff}{\ensuremath{\mathrm{d^2}}}	% Korrekt formatering af differential d. \diff
\newcommand{\dbar}{\ensuremath{\mathrm{\mathchar'26\mkern-12mu d}} \hspace{.15ex}}
\newcommand{\pt}[1]{\ensuremath{\left( #1 \right)}}


% Bløde d'er
\newcommand{\bd}[3][default]{\ifthenelse{\equal{#1}{default}}%
	{\frac{\partial#2}{\partial#3}}%
	{\left(\frac{\partial#2}{\partial#3}\right)_{\!#1}}}
\newcommand{\bdd}[3][default]{\ifthenelse{\equal{#1}{default}}%
	{\frac{\partial^2#2}{\partial^2#3}}%
	{\left(\frac{\partial^2#2}{\partial#3^2}\right)_{\!#1}}}
\newcommand{\bddd}[4][default]{\ifthenelse{\equal{#1}{default}}%
	{\frac{\partial^2#2}{\partial#3\,\partial#4}}%
	{\left(\frac{\partial^2#2}{\partial#3\,\partial#4}\right)_{\!#1}}}

% ==================================================================
% ============== Figurer og tabeller ===============================
% ==================================================================
\usepackage{graphicx}				% Håndtering af billeder
\usepackage{tabularx}				% Udvidede formateringsmuligheder
\usepackage{float}					% Udvidet styring af float-objekter bl.a [H] placement specifier
\usepackage{multirow}				% Fletning af celler
\usepackage{wrapfig}				% Wrap tekst om figurer
%\usepackage{sidecap}				% Figurtekst ved siden af figuren vha. SCfigure miljøet eller SCtable miljøet.
\usepackage{subcaption}				% Subfigures
\usepackage[textfont={rm},			% Fastsætter formatering af captions
			labelfont={bf},
			margin=6mm]
			{caption}
\usepackage{rotating}				% Kan bruges med \begin{sidewaysfigure}

\usepackage{dcolumn}
\hvisdansk{%
	\newcolumntype{d}[1]{D{,}{\mathord\mathcomma}{#1}}}{
	\newcolumntype{d}[1]{D{.}{.}{#1} }}
	% Giver mulighed for at lave decimaljusterede kolonner. OBS. \mathord\mathcomma bruges, da et almindeligt komma clasher med icomma pakken.

\usepackage[table]{xcolor}
\definecolor{tablegray}{gray}{0.9}
\definecolor{tablegreen}{RGB}{0,176,80}
\definecolor{tablered}{RGB}{255,50,50}
\definecolor{darkgray}{gray}{.4}

\newcommand{\NA}[1]{\multicolumn{1}{#1}{\textcolor{darkgray}{\text{--}}}	}		% Formatering af tom datacelle i tabel

\newcommand{\breakcell}[2][c]{%
  \begin{tabular}[#1]{@{}c@{}}#2\end{tabular}}

\usepackage{booktabs}				% Giver mulighed for ny formatering af tabeller.
\usepackage{colortbl}				% Udvidet farveformatering				

\usepackage{placeins}				% Bruges til \FloatBarrier

% Vil man spejlvende et billede kan det gøres ved at bruge \reflectbox{\includegraphics[...]{...}}

\setlength{\unitlength}{1cm}		% Fastsættelse af enheden til brug i picture-miljø
%\renewcommand{\arraystretch}{1.2}	% Øger lodret padding i tabeller
%\setlength{\tabcolsep}{2mm}		% Ændrer horisontal padding i tabeller

\usepackage{circuitikz}				% TikZ
\usepackage{pgfplots}
\newlength\figureheight	%størrelser der angives i matlab for at gøre figurer skalerbare.
\newlength\figurewidth
\usetikzlibrary{calc,intersections,shapes,decorations.pathreplacing,backgrounds,positioning,arrows,fit}

% ==================================================================
% ============== Punktopstilling ===================================
% ==================================================================
% Fjerner linjeafstand i punktopstillinger
\usepackage{enumitem}				% Udvidet formatering af itemize
	\setlist{noitemsep}				% Opsætning af opstilling (noitemsep kan erstattes med itemsep=<længde>
% OBS! Kan også fastsættes i den enkelte punktopstilling med \setlength{\itemsep}{<længde>}

% Formaterer nummereringen af enumerate
\renewcommand{\labelenumi}{\arabic{enumi}.}
\renewcommand{\labelenumii}{\labelenumi\arabic{enumii}.}
\renewcommand{\labelenumiii}{\labelenumii\arabic{enumiii}.}
\renewcommand{\labelenumiv}{\labelenumiii\arabic{enumiv}.}

% Formaterer symboler ved itemize
\renewcommand{\labelitemi}{$\bullet$}
\renewcommand{\labelitemii}{$\bullet$}
\renewcommand{\labelitemiii}{$\bullet$}
\renewcommand{\labelitemiv}{$\bullet$}


% ==================================================================
% ============== Bibliografi og indholdsfortegnelse ================
% ==================================================================

\RequirePackage{csquotes}
\usepackage[backend=biber%
			%,style=authoryear%		% Henvisninger vises med forfatter og år
			%,bibstyle%
			%,citestyle%
			,natbib=true%			% Giver natbib syntaks (\citet{}, \citep{}, \citep*{}...)
			,sorting=none%			% Kan også være bl.a. nty (name-title-year), nyt, ynt
			%,sortcase=false%		% Skal der sorteres Case Sensitive
			%,sortupper=false%		% Sorter først efter små bogstaver
			%,sortlos=bib%			% Hvordan skal List of Shorthands sorteres
			%,related=false%		% Skal der anvendes data fra relaterede entries
			%,sortcites=true%		% Hvis der citeres flere kilder samtidig, skal de så sorteres
			%,language=autobib%		% Sproget tilpasses efter babel (kan også være et bestemt sprog)
			%,block=space%			% Skal der indsættes ekstra afstand mellem større segmenter af en kilde
			%,hyperref=false%		% Fjerner hyperlinks fra cites
			%,backref=true,%			% Skriver i litteraturlisten, hvilke sider kilderne er citet på
			%,backrefstyle=two%		% Hvordan skal cites på efter hinanden følgende sider komprimeres i backref
			%,refsection=chapter%	% Skal referencer foregå i sektioner
			%,alldates=long%			% Hvordan skal datoer vises?
			%,isbn=false%			% Skal isbn printes
			%,url=false%			% Skal url printes
			,defernumbers=true%		% Hvis der køres med opdelte lister, vil første liste få de første numre
			]{biblatex}


% Overskriftens formatering
% Der kan tilføjes flere overskrifter og med forskellig formatering.			
\hvisdansk%
	{\newcommand{\bibhead}{Litteratur\label{litteratur}}}%
	{\newcommand{\bibhead}{Bibliography\label{litteratur}}}

%\renewcommand{\bibitemsep}{1cm}	Juster afstanden mellem kildeentries (der er mange andre afstande der kan justeres)

% Pdf-bogmærke til indholdsfortegnelsen
%\let\oldtableofcontents\tableofcontents
%\renewcommand{\tableofcontents}{%
%	\hvisdansk%
%		{\pdfbookmark{Indhold}{Indhold_bookmark}}%
%		{\pdfbookmark{Contents}{Indhold_bookmark}}%
%	\bookmarksetup{startatroot}%
%	\oldtableofcontents\FloatBarrier}


% ==================================================================
% ============== Arbejdsværktøjer ==================================
% ==================================================================
\hvisdansk{%
	\usepackage[danish]{todonotes}}{% Bruges til \todo og \missingfigure
	\usepackage{todonotes}}
	
\newcommand{\todosec}[2][]{\todo[inline, caption={2do}, #1]{
\begin{minipage}{\textwidth-4pt}#2\end{minipage}}}

\newcommand{\ts}{\textsuperscript}	% Kort kommando til opløftet tekst
	
\usepackage{comment}				% Udkommenter flere linjer ad gangen vha. comment miljø
\usepackage{blindtext}				% Opret fyldtekst med \blindtext og \Blindtext
%\usepackage{showframe}				% Viser rammer omkring de forskellige områder (god til at tjekke opsætning)
%\usepackage{showkeys}				% Ifm. referencer vises, hvilken label der refereres til

\usepackage{framed}








% ==================================================================
% ============== Bibtex (Legacy) ===================================
% ==================================================================

%\bibliographystyle{babplain}		% Vælger bibliografistil

%\usepackage[fixlanguage]{babelbib}	% Retter litteraturlisten til dansk skrivestil og danske ord
%\hvisdansk{\selectbiblanguage{danish}}{}		% Ellers ville der bl.a. stå and i stedet for og

% Indholdsfortegnelse
%\usepackage[nottoc,					% Fjerner indholdsfortegnelse fra indholdsfortegnelse
%			numbib,					% Giver litteraturlisten et kapitelnummer
%			]{tocbibind}
										
% Navngivning
%\hvisdansk{%
%		\settocname{Indholdsfortegnelse}	% Indholdsfortegnelsens navn (Hvorfor virker den ikke?)
%		\settocbibname{Litteraturliste}}{	% Litteraturlistens navn
%		\settocname{Table of Contents}
%		\settocbibname{Bibliography}}


% Laver indholdsfortegnelsen som section i stedet for chapter
%\makeatletter
%\renewcommand\tableofcontents{%
%    \section*{\contentsname
%            \@mkboth{%
 %          \MakeUppercase\contentsname}{\MakeUppercase\contentsname}}	% Laver \leftmark og \rightmark om til indhold
%           \MakeUppercase\contentsname}{}}	% Laver kun \leftmark om til indhold
%    \@starttoc{toc}%
%    } 
%\makeatother


% ==================================================================
% ============== Eksempler =========================================
% ==================================================================
\begin{comment}

\chead{
	\begin{picture}(15,1.8)
		\put(0,0){\includegraphics[width=15cm,clip=true,trim=34.5mm 260.5mm 28.4mm 22.3mm]{Figurer/Header/Header.pdf}}
		\put(0.05,0.15){\hyperref[chap:Introduktion]{\parbox[b][13mm][b]{19mm}{\ }}}
		\put(2.35,0.15){\hyperref[sec:indledning_til_baggrundsteori]{\parbox[b][13mm][b]{19mm}{\ }}}
		\put(4.6,0.15){\hyperref[chap:design]{\parbox[b][13mm][b]{22.5mm}{\ }}}
		\put(7.2,0.15){\hyperref[chap:beregninger]{\parbox[b][13mm][b]{19mm}{\ }}}
		\put(9.45,0.15){\hyperref[chap:simulering]{\parbox[b][13mm][b]{19.5mm}{\ }}}
		\put(11.7,0.15){\hyperref[chap:Forsoeg]{\parbox[b][13mm][b]{13mm}{\ }}}
		\put(13.35,0.15){\hyperref[chap:Afslutning]{\parbox[b][13mm][b]{16mm}{\ }}}
	\end{picture}}

% Sådan indsættes en figur i header med hyperlinks til forskellige kapitler

%---------------
\begin{lstlisting}[language = Matlab, tabsize = 3, label={lst:Matlab_script_timing}, caption = Matlab script der beregner $t_{delay}$ i \unit{ms} på baggrund af den udlæste værdi FartTid. De udregnede værdier gemmes i en .csv-fil. For alle heltalsværdier af FartTid fra 40 til 390 udregnes en tilsvarende værdi for $t_{delay}$.]
FartTid = 40:1:390; % Generer alle mulige værdier af FartTid fra 2-20 m/s
t_0 = zeros(1,length(FartTid)); % Generer tom liste til t_0 værdier
t_delay = zeros(1,length(t_0)); % Generer tom liste til t_delay værdier
f_timer = 8000000/1024; % Timerens clockfrekvens
v_reg = @(tid) -0.006782033821.*tid.^3 + 0.1496319837.*tid.^2 - 1.405270582.*tid + 7.397374966; % Funktionen v_reg
v_reg_int = @(tid) -0.006782033821/4.*tid.^4 + 0.1496319837/3.*tid.^3 - 1.405270582/2.*tid.^2 + 7.397374966.*tid; % Stamfunktionen til v_reg

 Bestem t_0 ud fra de mulige værdier af FartTid
for i = 1:1:(length(FartTid))
	for temp = -5.5:.01:20
		% Test om iterationen giver en lavere v_reg(t) end den målte fart
		if (.1*f_timer/FartTid(i) >= v_reg(temp))
			t_0(i) = temp; % Når grænsen findes, gemmes temps værdi i t_0
			break
		end
	end
end

 Beregn t_delay ud fra de mulige værdier af t_0
for i = 1:1:length(t_0)
	for temp = 0:0.00001:1
		% Test om iterationen giver en større strækning end 0,4 m
		if (v_reg_int(t_0(i)+temp+0.022) - v_reg_int(t_0(i)) >= 0.4)
			t_delay(i) = temp*1000; % temp omregnes til ms og gemmes i t_delay
			break
		end
	end
end

csvwrite('Affyringstiming.csv',t_delay); % Skriv data til csv-fil
\end{lstlisting}

% Sådan indsættes kode

%-------------------------
	\thispagestyle{empty}	
	\cleardoublepage
	\thispagestyle{fancy}
	\addtocounter{page}{-1}
	
% Sådan laves en blank bagside på et 2-sidet format.



\end{comment}