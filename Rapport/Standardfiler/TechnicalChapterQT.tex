\section{QT}
%short introduction to the chapter
Qt, pronounced "cute", is an open source cross-platform framework, mostly used for GUI(graphical user interface) programming. Qt has an easy to (re)use API(application programming interface), which in return gives high developer productivity. QT is C++ class library, hence new developers using Qt should have some understanding of C++.
This chapter introduces terminologies used in Qt, and tries to give some insight to how Qt operates and works regarding GUI development.  

%Maybe something about licencing and version used.


%somewhere here should the "simple" inheritance diagram be.

\subsection{Qt Class Hierarchy}
Qt broadly uses inheritance to create subclasses of instances in a natural way. QObject is the most basic class in Qt, see FIGURE. Alot of classes inherit from QObject, like QWidget, which is the base of all user interface objects. 
QObject -> The Qt Object Model -> features like below.
QObject -> Communication -> signals and slots. 
QObject -> object trees -> parenting system.

\subsection{The Meta-Object System}
QObject -> provides -> meta-object system
Extends c++ features
Q_OBJECT macro->mandatory->signal and slots.
The Meta-Object Compiler (moc) -> build process

\subsection{QWidgets}
events from system
widget -> parents
widget in a parent -> clipped how?
no parent -> top-level else child
composite widget -> many children
events -> QEvent eventhandling -> mousePressEvent() e.g.

\subsection{QMainWindow}
QMainWindow -> framework for application's user interface. REF TO FIGURE
QMenuBar -> pulldown menu items
QToolBar -> moveable panels
QDock -> docked or top-level window
Central Widget -> RWStudioView3D -> openGL

\subsection{QLayout}
Arrange child widgets within a widget
Updates when contents change
QHBoxLayout example. CODE AND PICTURE OF LAYOUT

\subsection{Signal and Slots}
Communication -> objects
Alternative to callback -> explain callback
Signal and slots -> connect() RET TO FIGURE
Signal -> 
Slot -> normal function (only special thing is it can be connected to a signal) -> found by moc


\subsection{Subclassing}

\subsection{Plug-in}

