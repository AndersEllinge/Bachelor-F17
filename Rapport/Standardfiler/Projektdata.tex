% ==================================================================
% ============== Titler og navne ===================================
% ==================================================================
\newcommand{\korttitel}{Bachelor}
\newcommand{\langtitel}{EasyInsert: Extended user interface and simplification of the interactions with RobWork}

\newcommand{\ForfatterAntal}{2}

\newcommand{\navnA}{Anders Ellinge}
\newcommand{\mailA}{aelli14@student.sdu.dk}

\newcommand{\navnB}{Mathias Elbæk Gregersen}
\newcommand{\mailB}{magre14@student.sdu.dk}

\newcommand{\navnC}{null}
\newcommand{\mailC}{null}

\newcommand{\navnD}{null}
\newcommand{\mailD}{null}

\newcommand{\navnE}{null}
\newcommand{\mailE}{null}

\newcommand{\navnF}{null}
\newcommand{\mailF}{null}

\newcommand{\afleveringsdato}{Sometime in 2017}


% Bruges til informationsside
\newcommand{\kursuskode}{RB-BAP6-U1}
\newcommand{\startdato}{1. of february}
\newcommand{\charactercount}{48,000}


\newcommand{\VejlederAntal}{1}

\newcommand{\vejlederA}{Lars-Peter Ellekilde}
\newcommand{\vejledermailA}{lpe@mmmi.sdu.dk}

\newcommand{\vejlederB}{Thomas Nicky Thulesen}
\newcommand{\vejledermailB}{tnt@mmmi.sdu.dk}

% ==================================================================
% ============== Metadata ==========================================
% ==================================================================											
% Pdf'ens metadata defineres (\AtBeginDocument kaldes, da dette ellers skulle defineres efter \korttitel m.m. er defineret)
\AtBeginDocument{				%OBS. Hvis hyperrefpakken udkommenteres giver dette fejl	
	\hypersetup{pdftitle={\korttitel}		% Metadata
				,pdfauthor={Anders Ellinge, Mathias Elbæk Gregersen}		% Metadata
				,pdfsubject={\langtitel}}}	% Metadata