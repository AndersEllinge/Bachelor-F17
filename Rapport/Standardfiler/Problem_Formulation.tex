\section{Problem Formulation}

RobWork is a collection of C++ libraries for simulation and control of robot systems. RobWork is used for research and education as well as for practical robot applications. It has been noted that there exists different tedious processes of editing the environment of the RobWork simulation, involving reconfiguring files, unintuitive user interaction or reloading software. If some of these processes can be shortened or made more intuitive, the user experience of RobWork would greatly benefit from it.
An example of an above-mentioned process: Adding a specific object to the environment of a simulated robot requires altering an .XML file and reloading RobWork.    
To specify goals for this bachelor project, the following features to implement is stated in order of importance:

\begin{itemize}
\item A user interface to initialize and load objects into the environment of RobWork with a simple drag and drop-like feature. 
\item Changing the configurations of the simulated robot in an intuitive matter, involving cursor interaction pulling the robot into the desired position.
\item Adding tools or end effector to a simulated robot in a drag and drop-like 
feature. The tool should then be able to snap unto various parts of the robot.
\end{itemize}

With skills and experience within software development, object oriented C++ and various programming oriented skills, the authors of this project will try to solve and implement the goals stated. The first step of the project would be to become familiar with RobWork and learn how the source code works. The next step is to design the software for the first feature and disclose a possible solution. The third step is to implement the first feature and test it.
 The success criteria for the project is a finished, tested and working feature for RobWork of the before mentioned first goal set for this project. If this criterion is met early, the project would continue in similar manner with the second and third feature.

\clearpage