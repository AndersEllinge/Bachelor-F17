\section{Use cases}
\label{app:useCases}
\subsection{Use cases for current use}
\textbf{Adding a frame/geometry to a WorkCell}

\noindent\textit{Main Success Scenario:} The user opens the XML-file describing the WorkCell in an editor. The user then writes the appropriate tag for adding the frame/geometry. The user then writes the appropriate tags for adding the additional information for the frame/geometry. The user then saves the WorkCell XML-file. The user then swaps to/ starts up RobWorkStudio. The user then loads the WorkCell from the XML-file via open in RobWorkStudio.\\

\noindent\textit{Alternate Scenarios:}

\noindent The user makes an error, the loader or parser catches this error and stops the loading process. The user is informed of the error.\\
\\

\noindent\textbf{Adding a device to a WorkCell}

\noindent\textit{Main Success Scenario:} The user opens the XML-file describing the WorkCell in an editor. The user uses the include tag to refer to the description of the device from another XML-file. The user then saves the WorkCell XML-file. The user then swaps to/ starts up RobWorkStudio. The user then loads the WorkCell from the XML-file via open in RobWorkStudio.\\

\noindent\textit{Alternate Scenarios:} 

\noindent The user manually writes the description of the device.\\

\noindent The user makes an error, the loader or parser catches this error and stops the loading process. The user is informed of the error. \\
\\

\noindent\textbf{Deleting an element}

\noindent\textit{Main Success Scenario:} The user opens the XML-file describing the WorkCell in an editor. The user deletes the tag that resembles the part the user wishes to delete. The user then saves the WorkCell XML-file. The user then swaps to/ starts up RobWorkStudio. The user then loads the WorkCell from the XML-file via open in RobWorkStudio.\\

\noindent\textit{Alternate Scenarios: }

\noindent The user makes an error, the loader or parser catches this error and stops the loading process. The user is informed of the error.\\

\clearpage
\subsection{Use cases for solution}

\noindent\textbf{Adding a frame/geometry to a WorkCell}

\noindent\textit{Main Success Scenario: }The user selects the wanted type of frame/geometry. The user then inputs the required information for the selected type of frame/geometry. The frame/geometry is then created and inserted into the WorkCell.\\

\noindent\textit{Alternate Scenarios:}

\noindent If the user supplied invalid information, the user is informed of the invalid information.\\
\\

\noindent\textbf{Adding a device to a WorkCell}

\noindent\textit{Main Success Scenario:} The user supplies the description of the device. The user then supplies the information about where the device should be placed and possibly a transform. The user also gives the device a name.\\

\noindent\textit{Alternate Scenarios:}

\noindent If the user gave the device a name that is already in use, the adding process is stopped and the user is informed of this error.\\

\noindent If the user supplied invalid information, the user is informed of the invalid information.\\
\\

\noindent\textbf{Deleting an element}
\noindent\textit{Main Success Scenario:} The user selects the element that the user wishes to delete. The user then initialises the deletion process.\\

\noindent\textit{Alternate Scenarios:}

\noindent If the user supplied invalid information, the user is informed of the invalid information.