\section{Conclusion/discussion of the project}
There are still errors and improvements to the solution, as mentioned in the chapters \ref{subsec:futureCreator}, \ref{subsec:futureLoader} and \ref{sec:futureOfThePlugin}. However testing the solution ourself, we were able to prove that the solution uphold the need to have requirements.\\

The need to have requirement 1 and 2 was satisfied with through the user interface and the creator, which proved capable of creating frames and geometries with the necessary parameters given in the requirement.
The need to have requirement 3 was satisfied with through the loader and the user interface, which was capable of loading devices from a XML file.
The need to have requirement 4 was satisfied with through the part of the solution described in section~\ref{sec:DelTab}. See appendix !!!!!, on the attached USB, for several videos showcasing/testing the solutions regarding the need to have requirements.\\

The solution made should be regarded as a proof of concept since it is not optimized in either speed or usability. In order to optimize for speed, a better understanding of coding C++ and the RobWork library is required. If better usability is to be achieved, having users test the solution and getting feedback regarding the usability would be optimal. The reason this testing phase was not done in this project is because of the time restraint of the project. In case of more time, this would be one of the things prioritised, since the core functionalities are somewhat in place.\\

In terms of the nice to have requirements, it is our understanding that the user interface is intuitive to use. The functionality for the user to define a library for the devices were also made. Albeit a simple solution, its simplicity gives the solution, for defining a library, modularity. The nice to have requirement 3, regarding an undo action, was not implemented due to the time constraint imposed on the project. It was also discovered that this functionality would be rather difficult to implement.\\

All in all, the project was considered a success.